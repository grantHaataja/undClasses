%\title{Math488-2019-PS!-Prob3}
\documentclass[letterpaper,12pt]{amsart}
 
\usepackage[margin=1in]{geometry} 
\usepackage{amsmath,amsthm,amssymb}
\usepackage{enumitem}
 
\newcommand{\N}{\mathbb{N}}
\newcommand{\Z}{\mathbb{Z}}
 
\newenvironment{theorem}[2][Theorem]{\begin{trivlist}
\item[\hskip \labelsep {\bfseries #1}\hskip \labelsep {\bfseries #2.}]}{\end{trivlist}}
\newenvironment{lemma}[2][Lemma]{\begin{trivlist}
\item[\hskip \labelsep {\bfseries #1}\hskip \labelsep {\bfseries #2.}]}{\end{trivlist}}
\newenvironment{exercise}[2][Exercise]{\begin{trivlist}
\item[\hskip \labelsep {\bfseries #1}\hskip \labelsep {\bfseries #2.}]}{\end{trivlist}}
\newenvironment{problem}[2][Problem]{\begin{trivlist}
\item[\hskip \labelsep {\bfseries #1}\hskip \labelsep {\bfseries #2.}]}{\end{trivlist}}
\newenvironment{reflection}[2][Reflection]{\begin{trivlist}
\item[\hskip \labelsep {\bfseries #1}\hskip \labelsep {\bfseries #2.}]}{\end{trivlist}}
\newenvironment{proposition}[2][Proposition]{\begin{trivlist}
\item[\hskip \labelsep {\bfseries #1}\hskip \labelsep {\bfseries #2.}]}{\end{trivlist}}
\newenvironment{corollary}[2][Corollary]{\begin{trivlist}
\item[\hskip \labelsep {\bfseries #1}\hskip \labelsep {\bfseries #2.}]}{\end{trivlist}}

\newenvironment{solution}
  {\begin{proof}[Solution]}
  {\end{proof}}
 
\begin{document}
 
% --------------------------------------------------------------
%                         Start here
% --------------------------------------------------------------
 
%\renewcommand{\qedsymbol}{\filledbox}
 
\title{GRE Problem Set $\#1$}%replace "1" with the appropriate number
\author{Ima Student\\ %replace with your name
MATH $488$ - Mathematics Capstone - Fall 2019} 
 
\maketitle
 
\begin{problem}{2}
The value of the integral 
   $\displaystyle \int_0^1\sqrt{e^{2x}+e^{-2x}+2} \;dx$ is
 
    \vspace*{0.15cm}
    \begin{enumerate}[itemsep=0.15cm]
      \item $e-\frac{1}{e}$
      \item $e+\frac{1}{e}$
      \item $e-\frac{1}{e}+1$
      \item $e+\frac{1}{e}-1$
      \item $2e$
    \end{enumerate}
\end{problem}
\begin{solution}  %% Type your solution/proof between \begin{solution} and \end{solution} %%

\end{solution}
% --------------------------------------------------------------
%     You don't have to mess with anything below this line.
% --------------------------------------------------------------
 
\end{document}

%%%%%%%%%%%%%%%%%%%%%%%%  The following is the original LaTeX for the problem set %%%%%%%%%%%%%%%%
%%%%%%%%%%%%%%%%%%%%%%%% You could copy and paste a problem statement from here to %%%%%%%%%%%%%%%
%%%%%%%%%%%%%%%%%%%%%%%% replace the problem statement I've given you above.       %%%%%%%%%%%%%%%

% \begin{document}
% \noindent
% \textbf{Math 488 Fall 2018 \hfill GRE Problem \#1 \hfill Name:\hbox to 2in{\hrulefill}}

% \vspace*{0.25cm}
% \begin{enumerate}[leftmargin=*,itemsep=0.5cm]
%   \item Evaluate $\displaystyle \int_{-2}^{-1} \frac{1}{\sqrt{-x^2-6x}}\;dx$.
  
%     \vspace*{0.15cm}
%     \begin{enumerate}[itemsep=0.15cm] 
%       \item $\arcsin(2/3)-\arcsin(1/3)$
%       \item $\arcsin(2/3)-\arccos(1/3)$
%       \item $\arctan(2/3)-\arctan(2/3)$
%       \item $\arctan(2/3)+\arctan(2/3)$
%       \item $\arccos(2/3)+\arccos(1/3)$
%     \end{enumerate}
    
%  \item Let $f(x)=x^3+6x^2-32$. Define $T_{c,f}(x)=\sum_{n=0}^\infty a_n(x-c)^n$ to be
%  the Taylor series for the function $f$ centered at $c$. For what values of $C$ does
%  $a_0=0$?
 
%  \vspace*{0.15cm}
%  \begin{enumerate}[itemsep=0.15cm]
%    \item $c=3,2$
%    \item $c=-3,2,-2$
%    \item $c=4,2$
%    \item $c=-4,2,-2$
%    \item $c=-4,2$
%   \end{enumerate}
  
%   \item The value of the integral $\displaystyle \int_0^1\sqrt{e^{2x}+e^{-2x}+2} \;dx$ is
  
%     \vspace*{0.15cm}
%     \begin{enumerate}[itemsep=0.15cm]
%       \item $e-\frac{1}{e}$
%       \item $e+\frac{1}{e}$
%       \item $e-\frac{1}{e}+1$
%       \item $e+\frac{1}{e}-1$
%       \item $2e$
%     \end{enumerate}
 
%  \item For the function
%  \[
%   f(x)=\begin{cases}
%           \frac{x^2y}{x^4+y^2}, &(x,y)\not=(0,0) \\
%           0, & (x,y)=(0,0)
%        \end{cases}
%  \] which of the following is true?
  
%  \vspace*{0.15cm}
%  \begin{enumerate}[itemsep=0.15cm,label=\Roman*]
%    \item $f$ is not continuous at $(0,0)$.
%    \item $f$ is differentiable everywhere.
%    \item $f$ has well-defined partial derivatives everywhere 
%          (i.e.~$\frac{\partial f}{\partial x}, \frac{\partial f}{\partial y}$
%          are both defined)
%    \item $f$ is continuous at $(0,0)$ but not differentiable at $(0,0)$
   
    
%  \vspace*{0.15cm}
%  \begin{enumerate}[itemsep=0.15cm]
%    \item I only
%    \item II only
%    \item III only
%    \item IV only
%    \item I and III only.
%   \end{enumerate}
%   \end{enumerate}  
  
%   \item Evaluate the limit: $\displaystyle \lim_{n\to\infty}\left(3^n+5^n\right)^{2/n}$.
%     \vspace*{0.15cm}
%  \begin{enumerate}[itemsep=0.15cm]
%    \item $0$
%    \item $1$
%    \item $9$
%    \item $16$
%    \item $25$
%   \end{enumerate}
  
%  \item Evaluate $\displaystyle \int_0^{\pi/2} \frac{1}{1+\tan(x)}\;dx$ (TRICKS INVOLVED!)
  
%   \vspace*{0.15cm}
%  \begin{enumerate}[itemsep=0.15cm]
%    \item $\frac{\pi}{2}$
%    \item $\frac{\pi}{3}$
%    \item $\frac{\pi}{4}$
%    \item $\frac{\pi^2}{2}$
%    \item $\frac{\pi^2}{4}$
%   \end{enumerate}
  
%  \item The series $1-1/3+1/5-1/7+1/9-1/11+\cdots$ converges to:\newline
%         [Hint: Start playing around with 
%                $\frac{1}{1-x}=\sum_{n=0}^\infty x^n$.] 
 
%  \vspace*{0.15cm}
%  \begin{enumerate}[itemsep=0.15cm]
%    \item $\frac{\pi}{6}$
%    \item $\frac{\pi}{5}$
%    \item $\frac{\pi}{4}$
%    \item $\frac{\pi^2}{2}$
%   \end{enumerate}
% \end{enumerate}
% \end{document}
