\documentclass[11pt,final,twoside]{article} %setup
%set margins and load symbols and such
\usepackage[top=20mm,inner=20mm,outer=30mm,bottom=20mm,
marginparwidth=23mm,marginparsep=4mm]{geometry}
%header of document
\title{CSci242, Spring 2019\\
	\texttt{Assignment 3\\
	\texttt{Chapter 11.5 Exercise R-11.1}}}
\author{Grant Haataja}
\date{\today}
%set paragraph spacing
\linespread{0.8}

%make stuff happen
\begin{document}
	\maketitle

\begin{enumerate}
	\item Characterize each of the following recurrence equations using the master theorem
	(assuming that $T(n) = c$ for $n < d$, for constants $c > 0$ and $d \geq 1$).
	\begin{enumerate}
		\item $T(n) = 2T(n/2) + \log n$
		\item $T(n) = 8T(n/2) + n^2$
		\item $T(n) = 16T(n/2) + (n\log n)^4$
		\item $T(n) = 7T(n/3) + n$
		\item $T(n) = 9T(n/3) + n^3\log n)$ 	
	\end{enumerate}
	\item Answers:
	\begin{enumerate}
		\item Case 1 of the master theorem applies. $n^{\log_ba} = n^{\log_22} = n$ and $f(n) = \log n$ so $f(n)$ is small compared to $n^{\log_ba}$. $T(n) \sim n$.
		\item Case 1 applies. $T(n) \sim n^3$.
		\item Case 2 applies. $T(n) \sim n^4(\log n)^5$.
		\item Case 1 applies. $T(n) \sim n^{1.77...}$.
		\item None of the cases apply, so $T(n) \sim n^3\log n$.
	\end{enumerate}
\end{enumerate}
	
\end{document}