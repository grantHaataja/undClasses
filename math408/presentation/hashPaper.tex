\documentclass{article}

\usepackage{mathtools}
\usepackage[top=20mm,inner=20mm,outer=30mm,bottom=20mm,
marginparwidth=23mm,marginparsep=4mm]{geometry}
%header of document
\title{Cryptographic Hash Functions}
\author{Grant Haataja}
\date{\today}
%set paragraph spacing
\linespread{1.5}

\begin{document}
	\maketitle	
%table of contents
	\tableofcontents
	
	%start the hack
Hash functions are an important facet of cybersecurity.  



	A hash function is a mathematical algorithm that takes input of arbitrary length and maps it through a series of transformations to a string of fixed length.

{Requirements of Hash Functions}
		Computationally efficient
		Pre-Image Resistance 
		Deterministic  
		Collision Resistance
		Outputs should have the appearance of complete randomness

{How do Hash Functions Work?}
		Hash functions take an input of any size and go through a series of steps to change it into something theoretically unrecognizable to the original message
		The output of a hash function is commonly referred to as a \textit{digest}.
	
	A hash function can be as simple as $$f(x) = x\mod7.$$
		
	Then any input could be converted into a decimal number and sent through the function. For example, consider the input: "Message". If we apply the standard conversion from letters to numbers, (A $\rightarrow$ 1,B $\rightarrow$ 2,...,Z $\rightarrow$ 26), "Message" becomes 13 5 18 18 1 7 5. Then, we take it as the number 1,351,818,175 and send it through the function, giving $$f(1351818175) = 1351818175 \mod 7 = 1$$.

{Uses of Cryptographic Hash Functions}
	\begin{itemize}
		\item Verification of files or messages
		\item Password protection
		\item file or object identifier over file-sharing networks
		\item Bitcoin mining
		\item Bitcoin creation of addresses\newline
		
	\end{itemize}

{Types of Cryptographic Hash Functions}
	\begin{itemize}
		\item MD4 (Message Digest Algorithm )
		\item MD5 
		\item SHA-1 (Secure Hashing Algorithm)
		\item SHA-2
		\item SHA-3 
		\item RIPEMD-160 
		\item CryptoNight
		\item PBKDF2
		\item bcrypt
		\item Argon2\newline
	\end{itemize}

{MD5 (Message Digest Algorithm 5)}
	\begin{itemize}
		\item 128-bit hash
		\item Created in 1991 by Ronald Rivest as a replacement for MD4
		\item Collision resistance broken in 2004 after roughly $2^{21}$ hashes, reportedly taking only an hour to complete
		\item Still used widely but no longer secure for sensitive data
		\item Common passwords can be found by typing their hashes into Google search\newline
	\end{itemize}

{SHA-256 (Secure Hashing Algorithm)}
	\begin{itemize}
		\item 256-bit hash
		\item Designed by the NSA as part of the SHA-2 set in 2001
		\item Collision resistance hasn't been fully broken yet
		\item Most commonly used and trusted hashing algorithm currently, although industry is in the process of switching to SHA-3\newline
	\end{itemize}

{Hashing Examples}
	\begin{itemize}
		\item We have the message "Hashing is a secure way to store passwords, change my mind".
		\item MD5: b30d5d9bc11f392daa4950be13d35106
		\item SHA-256: a1b196e3e571170f5111c9f0058d20cc4761546468fbe6bdc16f2663f6096e7a
		\item Now we will remove the comma from the message and hash "Hashing is a secure way to store passwords change my mind".
		\item MD5: 91a6526d84520f2aad7f25116880d4e9
		\item SHA-256 7809dace6f2004d72611a4a59f2d6fa5d2ab0307c2b7de741b86779275681a5b\newline
	\end{itemize}

{Pros and Cons of Hashing}
	\begin{itemize}
		\item PROS
		\begin{itemize}
			\item Portable and easy for storing
			\item Simple and fairly secure way to store sensitive data to be used for verification, as long as the hashes are salted
			\item Does not require any key for verification 
		\end{itemize}
		\item CONS
		\begin{itemize}
			\item Cannot be used to send messages, since there is no way to reverse secure hashes
			\item Unsalted hashes can become security risks   \newline
		\end{itemize}
		
	\end{itemize}

{Sources}
\begin{thebibliography}{}
	\bibitem{Cryptographic Hash Functions Explained: A Beginner's Guide}
	Daniel. (2018, December 03). Cryptographic Hash Functions Explained: A Beginner's Guide. Retrieved from {https://komodoplatform.com/cryptographic-hash-function/}
	
	\bibitem{Tutorials Point Website}
%	Tutorialspoint.com. (2019). Cryptography Hash functions. Retrieved from https://www.tutorialspoint.com/cryptography/cryptography_hash_functions.htm
	
	\bibitem{Practical Cryptography Website}	
%	Lyons, J. (2012). MD5 Hash. Retrieved from \url{http://practicalcryptography.com/hashes/md5-hash/}
	
	\bibitem{Cryptographic Hash Functions}
%	Holbreich, A. (2016, November 01). Cryptographic Hash Functions. Retrieved from \url{http://alexander.holbreich.org/cryptographic-hash-functions/}	
	
	\bibitem{Blockgeeks Website}
	Mikoss, I. (2019). The In's and Outs of Cryptographic Hash Functions (Blockgeek's Guide).
%	Retrieved from \url{https://blockgeeks.com/guides/cryptographic-hash-functions/}
\end{thebibliography}

\end{document}